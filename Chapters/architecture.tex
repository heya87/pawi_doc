% Chapter Template

\chapter{Software Architecture} % Main chapter title

\label{architecture} % Change X to a consecutive number; for referencing this chapter elsewhere, use \ref{ChapterX}

\lhead{\emph{Software Architecture}} % Change X to a consecutive number; this is for the header on each page - perhaps a shortened title

%----------------------------------------------------------------------------------------
%	SECTION 1
%----------------------------------------------------------------------------------------

\section{Version}

\begin{tabular}{| p{1.5cm} | p{2cm} | p{9cm} | p{1.5cm} |}
	\hline
	Version & Date 		& Change & Author \\ \hline
	0.1 	& 8.10.2014 		& Setup document  										& JR \\ \hline
	0.2 	& 12.10.2014		& Class diagrams										& JR \\ \hline
	0.3 	& 16.10.2014		& Text													& JR \\ \hline
	0.4 	& 20.10.2014		& Activity diagram										& JR \\ \hline
	1.0 	& 21.10.2014		& Grammar, Diagram fixes								& JR \\ \hline
	1.1 	& 22.10.2014		& Adding description of all business logic classes, approach and extended introduction & JR \\ \hline
	1.2 	& 23.10.2014		& Grammar & JR,LR \\ \hline
	1.3 	& 16.12.2014		& Grammar & JR \\ \hline

\end{tabular}

\section{Introduction} 

This document describes the software architecture of the context extraction test framework. 
The context extraction test framework will perform automated text extraction on a set of HTML test data with two to three different text extraction algorithms. After measuring the performance of each algorithm, an output file with the measured results is generated.

\section{System Overview}

This section describes the approach for elaborating the software architecture and gives an overview over the software architecture. 
The following figure from the software requirement specification is the basis for elaborating the software architecture.

%\begin{landscape}
\ref{asfdsafdsa}
\begin{figure}
\includegraphics[width=15cm]{Figures/AppOverview.pdf}
\caption{blablalb}
\label{asfdsafdsa}
\end{figure}
%\end{landscape}

First all possible entities in the domain are identified and a data model is elaborated. The data model is described in section \ref{subsec:Data model}. Then the business logic is researched based on the figure above. With this knowledge, the five packages main, reader, classifier, analyzer and writer are defined. The functions for each package is then evaluated and split into single classes. During the implementation of the first approach, some classes are adapted due to unforeseen circumstances. The outcome is described in section \ref{subsec:Business Logic}.
\pagebreak

\section{Logical view}

This section describes how the system is structured in terms of units of implementation.


\subsection{Data model}
\label{subsec:Data model}

The following diagram shows the data model of the application. 

\includegraphics[width=14cm]{Figures/dataModel.pdf}


\subsection{TestCase}
A TestCase object is generated for each HTML/content file pair in the input folders. A TestCase has a unique  name that also matches the name of the content and HTML file.

\subsection{HtmlFile}
Each TestCase has a HtmlFile object. It contains the content of the actual HTML file as a String and the file path.

\subsection{ContentFile}
Each TestCase has a ContentFile object. It contains the content of the actual text file as a String and the file path.

\subsection{ExtractedContent}
Each TestCase can have multiple ExtractedContent objects. Each of them represents a result of a content extraction from an extractor such as JusText or Boilerpipe. It contains the extracted text as well as the extracted text separated in Block objects with the generated classification.

\subsection{Result}
A TestCase or an ExtractedContent object can have Result objects. The results are key value pairs which represent analytical data. An example for a Result related to a TestCase would be the word count of the content file, which is generally valid. An example for a Result related to an ExtractedContent would be the word count of true negative words, which is only valid for one specific ExtractedContent.

\subsection{Config}
The Config object contains all configurable information about a test run. It is generated from the config.txt file at the beginning of the test. Configuration parameters like test language, used extractors, possible filters are defined by the Config object.
\begin{landscape}

\subsection{Business Logic}
\label{subsec:Business Logic}

The following diagram shows the business logic of the application. The diagram does not show all of the classes but the most important ones. 

\includegraphics[width=23cm]{Figures/architecutre.pdf}


\end{landscape}



\pagebreak
\subsection{Description of single classes}

\begin{longtable}{p{4cm}|p{2cm}|p{8cm}}
\hline
\textbf{Class} &
\textbf{Package} &
\textbf{Description}
 \\ \hline
TestManager &
testManager &
The TestManager class manages the whole business logic that manages TestCase objects through the whole test process from reading the file content to writing the test results into an output file.
 \\ \hline
FileObserver &
reader &
The FileObserver class checks the HTML and content directory for files of the same name and creates TestCases from each found pair. The folders are checked with the FileReader class and the content of the files are read with the StringReader class.
\\ \hline
StringReader &
reader &
The StringReader class reads a text file and returns the content as String. The class is made for easier mocking of the BufferedReader so that testing of other classes which are dependent on external files becomes much easier.
\\ \hline
FileReader &
reader &
The FileReader class returns a File objects for each found file in a directory given by a parameter.
\\ \hline
ExtractorManager &
classifier &
The ExtractorManager manages all available Extractors. Each extractor used for a test must be initialized in this class and added to the ExtractorList. Each TestCase is then extracted by every IExtractor in the ExtractorList.
\\ \hline
IExtractor &
classifier &
The IExtractor is the interface to the different extractor. The interface is very lightweight. The parameter is the text that should be extracted and the return value is the extracted text.
\\ \hline
BoilerpipeExtractor &
classifier &
The BoilerpipeExtractor implements the IExtractor and is the interface to the Boilerpipe package. It handles all dependencies on the Boilerpipe package and returns the extracted content as a string.
\\ \hline
JusTextExtractor &
classifier &
The JusTextExtractor implements the IExtractor and is the interface to the JusText python program. The Java ProcessBuilder is used to create operating processes. One can then execute operating system commands and run the python script. The JusText python script creates a text file with the extracted content which is read by the JusTextExtractor class and returned as String.
\\ \hline
AnalyzerManager &          
analyzer &          
The AnalyzerManager manages all available analyzers. Each analyzer which is used for the actual test must be initialized in this class and added to the AnalyzerList. Each TestCase is then analyzed by every IAnalyzer in the AnalyzerList.
\\ \hline
IAnalyzer &
analyzer &
The IAnalyzer interface is a simple interface to the different analyzers. Each analyzer can generate one or more Result objects.
\\ \hline
DiffToolAnalyzer &
analyzer &
The DiffToolAnalyzer compares two Strings and generates the values TP, FP, TN and FN and puts them as Result object into the test cases.
\\ \hline
StatisticsAnalyzer &
analyzer &
The StatisticsAnalyzer extracts the values TP, TN, FP, FN from the test cases and calculates Precision, Recall, F-Measure, Fallout and accuracy and FN and puts them as Result object into the test cases.
\\ \hline
ResultFilter &
writer &
Filters all TestCases for the filters given in the configuration objects.
\\ \hline
Formatter &           
writer &           
The Formatter class formats the Result from a test run as String in a CSV file structure. This means that a table with the Result keys as table header for each column is created. For each TestCase a row with the Result value field as values is added to the table. The outcome is a CSV file which can easily be imported into another program such as Excel so that one can work with the data.
\\ \hline
WriterWrapper &
writer &
The writer wrapper writes a string into a text file. It is used to wrap the BufferedWriter so that mocking and testing of dependent classes is easier.
\\ \hline
ConfigReader &
reader &
The ConfigReader class reads the config.txt file and generates a Config object. This object contains the test language, the used extractors, possible filters and single test cases which are inspected for closer inspection.
\\ \hline
\end{longtable}


\section{Process view}

This section describes the dynamic aspects of the application.

\includegraphics[width=15cm]{Figures/activityDiagram.pdf}

The data model is passed through the business logic and is enriched with extracton results and statistical data during the test procedure. First the input directories are checked for files by the FileObserver and TestCases are generated for each file pair with the same name. Then all the TestCases are handed over to the ExtractorManager. The ExtractorManager extracts each TestCases with all available implementations of IExtractor and puts the Results into ExtractionResults. After that, all the TestCases are handed over to the Analyzer which runs each implementation of IAnalyzer. Each IAnalyzer produces at least one Result and puts it into the TestCase. To simplify the diagram, only two Analyzers are drawn. After generating the Results, they are filtered if any filter is set in the Config object. Then, the Formatter serializes the Result Objects into a String of CSV data and the WriterWrapper persists the CSV data into an output file.


\section{Development view}
\label{architecture:developmentView}

This chapter describes the used tools and libraries during the development process as well as the continuous integration process.


\subsection{Continuous Integration Process}

The continuous integration process for this process is handled with the version control tool Git, the continuous integration server Travis CI and the build automation tool Gradle. The individual tools are described in more details in the following sections. Since this is a one man project, there is only one Git branch present for most time of the development phase. The application is developed on the local branch and unit tests are written for critical components. The application is then built locally with Gradle. Changes are pushed to the master branch of the Git repository. As soon as there are any new commits on the master repository, the application is built on Travis CI too. If the build is not successful. An error report is sent to the user.

\subsection{Git}

Git is a distributed version control system. Unlike other version control systems like Subversion, Git is not using a central server but each user has his own copy with the complete history on the local system. It is much easier to work with additional branches or tags. 
Because of these advantages and because Layzapp is working with it as well, Git was chosen as SCM for this project for the source code and the documentation.
The project is open source and available under following links.

\begin{itemize}
\item code: \url{https://Github.com/heya87/pawiTwo}
\item documentation: \url{https://Github.com/heya87/pawi_doc}
\end{itemize}


\subsection{Gradle}

Gradle is a project automation tool which uses a Groovy-based domain-specific language (DSL) instead of the more traditional XML form of declaring the project configuration. All the dependencies of the project are handled with Gradle and it is very easy to deploy on a new system. The user only needs to download the Git repository and can build the project with the Gradle wrapper without installing any new software. All dependencies are then downloaded automatically and the user can start working without caring about missing packages.
The build process is defined in the build.gradle file which is located in the source directory of the project. 

\subsection{Travis CI}

Travis CI is an open source build server. It is easy to use in combination with Git. One only needs to add a configuration file in the source directory of the project and define the Git repository. Afterwards the project is build for every change. If the build does not pass, a mail is sent to the user. The actual status of the build can be found under following link \url{https://travis-ci.org/heya87/pawiTwo}.

\subsection{JusText}

An implementation of the JusText algorithm is available in Python. The resources are available under following link \url{https://code.google.com/p/justext/}. There is no implementation in Java so for the first approach the python application is called from Java with a ProcessBuilder.
The command to extract an HTML file with JusText is very simple:

\begin {lstlisting}
justext -s English /path/page.html > cleaned-page.txt
\end{lstlisting}

This command extracts the HTMl file page.html into a text file called cleaned-page.txt. The Java ProcessBuilder performs system commands as they are used in a console and can handle the outcome if needed. For the project there is no return values needed. The needed content is the generated text file which is then read and processed for further use.

\subsection{Boilerpipe}

An implementation of the Boilerpipe Algorithm is available in Java. The resources are available under following link \url{https://code.google.com/p/boilerpipe/}. This algorithm can be used out of the box with calls against the Java API.

\subsection{diff-match-patch}

The Java library diff-match-patch \cite{google:diffMatchPatch} is used to compare text files and find differences. For this file, it is used to extract the values TP, TN, FP, FN when comparing the actual content files with the outcome of the algorithms.


\section{Physical view}

Following diagram shows the physical view of the application. The application is running on a single device, which results in a very simple diagram. The external components are the Boilerpipe algorithm implemented in Java and the JusText algorithm implemented in Python. The python program is called with system calls. All the components are located on the same device. 
The config.txt file contains configuration parameter for the application and the different output text files contain the results of the test framework.

\includegraphics[width=10cm]{Figures/deploymentDiagram.pdf}


