% Chapter Template



\chapter{Planning} % Main chapter title

\label{Planning} % Change X to a consecutive number; for referencing this chapter elsewhere, use \ref{ChapterX}

\lhead{Planning \emph{}} % Change X to a consecutive number; this is for the header on each page - perhaps a shortened title

%----------------------------------------------------------------------------------------
%	SECTION 1
%----------------------------------------------------------------------------------------

\section{Version}

\begin{tabular}{| p{1.5cm} | p{2cm} | p{9cm} | p{1.5cm} |}
    \hline
    Version & Date      & Change & Author \\ \hline
    0.1     & 15.09.2014        & Setup document                                        & JR \\ \hline
    1.0     & 28.09.2014        & Draft planning                                        & JR \\ \hline
    1.1     & 18.09.2014        & Adding overview                                       & JR \\ \hline
    1.2     & 6.10.2014        & Stories for MS2                                       & JR \\ \hline
    1.3     & 23.10.2014        & Stories for MS3                                      & JR \\ \hline
    1.4     & 7.11.2014        & Stories for MS4                                       & JR \\ \hline
    1.5     & 20.10.2014        & Stories for MS5                                       & JR \\ \hline
    1.6       & 8.12.2014        & Adding time tracking chapter                      & JR \\ \hline
    1.7       & 8.12.2014        & Grammar                                            & JR \\ \hline


\end{tabular}

\section{Planning concept}

To plan the project, the planning framework scrum was used.\\
For a first rough planning, the assignment is split into working packages and assigned to milestones. Delivery objects are defined for each milestone.\\
This plan is then assigned to the given time table of about 12 weeks. The project effort is defined as 180 hours. This results in about 15 hours of work per week.

As working with Scrum, each milestone was defined as a sprint. At each start of a sprint, the working packages were adjusted if needed and were then split into stories. In the following section all the milestones are listed with their working packages and stories. The working packages were defined at the very beginning of the project and the stories were defined at the start of each milestone/sprint. To see the adjustments made during the project, the definition of the working packages was not adapted for this final documentation.
 

\section{Milestones overview}

	\begin{tabular}{ | p{3cm} | p{2cm} | p{2cm} | p{2cm} | p{2cm} | p{2cm} |}
	\hline
	\textbf{Name} & \textbf{Shortcut} & \textbf{Weeks} & \textbf{Estimated hours} & \textbf{Hours  \mbox{total}} & \textbf{Closing date} \\ \hline
	Milestone one & m1 & 2.5 & 39 & 39 & 01.10.2014 \\
	Milestone two & m2 & 3 & 45 & 84 & 22.10.2014 \\
	Milestone three & m3 & 2 & 30 & 114 & 05.11.2014 \\
	Milestone four & m4 & 2 & 30 & 144 & 19.11.2014 \\
	Milestone five & m5 & 2.5 & 38 & 182 & 08.12.2014 \\
	\hline
	\end{tabular}


\section{Delivery objects}

    \begin{tabular}{ | p{2.5cm} | p{2.5cm} | p{9cm} |}
    \hline
    \textbf{Milestone} & \textbf{Delivery date} & \textbf{Delivery objects}\\ \hline
    Milestone one & 01.10.2014 &
        \begin{itemize}
        \item System specification
        \item Sketch software architecture
        \item Short presentation CI environment
        \item Draft risk evaluation
        \end{itemize}\\
    \hline
        Milestone two & 22.10.2014 &
        \begin{itemize}
        \item Elaborated software architecture
        \item Tested code of test framework (tbd: which components)
        \item Interface definition for justext/boilerplate components
        \item HTML test data
        \end{itemize}\\
    \hline
        Milestone three & 05.11.2014 &
        \begin{itemize}
        \item Working test environment with both justext and boilerplate components integrated
        \end{itemize}\\
    \hline
        Milestone four & 19.11.2014 &
        \begin{itemize}
        \item Evaluation environment for output data of test framework
        \item First approach to new algorithm
        \end{itemize}\\
    \hline
        Milestone five & 08.12.2014 &
        \begin{itemize}
        \item Implementation of new algorithm
        \item Final documentation
        \item Final presentation
        \end{itemize}\\
    \hline
    \end{tabular}


\section{Milestone one - m1}

\begin{itemize}
\item Closing date date: 1.10.2014
\item Available time: ca. 39h
\end{itemize}

    \begin{tabular}{ | p{10cm} | p{2cm} | p{2cm} |}
    \hline
    \textbf{Working package} & \textbf{Shortcut}& \textbf{Estimated time} \\ \hline
    Planning & s1 &4h \\
    Research HTML / Algorithms & s2 & 8h \\
    System specification & s3 & 12h \\
    Risk evaluation & s4 & 3h\\
    Draft software architecture & s5 & 8h \\ 
    Configuration CI environment & s6 & 4h \\ \hline
    Total & & 39h\\
    \hline
    \end{tabular}

\subsection {Stories m1}

    \begin{tabular}{ | p{4cm} | p{10cm} |}
    \hline
    \textbf{Title} & Planning\\ \hline
    \textbf{Id} & s0\\ \hline
    \textbf{Estimated time} & 4h \\ \hline
    \textbf{Description} &  As a project owner, you need to have a time schedule so you can see when you will achieve which results. The PAWI project is split into several working packages which are then split into single stories. The working packages are assigned to milestones and for each milestone, delivery objects are defined. This can be a document, a piece of test or production code or some other kind of work.\\
    \hline
    \end{tabular} \\\\

    \begin{tabular}{ | p{4cm} | p{10cm} |}
    \hline
    \textbf{Title} & Research HTML / Algorithms\\ \hline
    \textbf{Id} & s1\\ \hline
    \textbf{Estimated time} & 8h \\ \hline
    \textbf{Description} & My knowledge of HTML and content extraction algorithms is still limited. In order to find out what challenges I will face and which aspects I will have to take into consideration for performing the first tasks, a short research on these topics is needed.\\
    \hline
    \end{tabular} \\\\


    \begin{tabular}{ | p{4cm} | p{10cm} |}
    \hline
    \textbf{Title} & System specification\\ \hline
    \textbf{Id} & s2\\ \hline
    \textbf{Estimated time} & 12h \\ \hline
    \textbf{Description} & The PAWI project is defined through a short project description. This description does not cover all necessary information to both plan and perform this project. The key features, interfaces and delivered objects have to be defined more closely. The system specification should cover all these requirements.\\
    \hline
    \end{tabular} \\\\


    \begin{tabular}{ | p{4cm} | p{10cm} |}
    \hline
    \textbf{Title} & Draft software architecture\\ \hline
    \textbf{Id} & s3\\ \hline
    \textbf{Estimated time} & 8h \\ \hline
    \textbf{Description} & A first rough software architecture should be made as soon as possible, so that any misunderstandings between tutors and student can be uncovered. Moreover, it is easier to plan further steps when the software is split into several parts.\\
    \hline
    \end{tabular} \\\\


    \begin{tabular}{ | p{4cm} | p{10cm} |}
    \hline
    \textbf{Title} & Risk evaluation\\ \hline
    \textbf{Id} & s4\\ \hline
    \textbf{Estimated time} & 8h \\ \hline
    \textbf{Description} & Potential risks should be uncovered with the knowledge that was gathered by defining the specification and the software architecture. Further actions can be defined to minimize the above mentioned risks.\\
    \hline
    \end{tabular}
    

    \begin{tabular}{ | p{4cm} | p{10cm} |}
    \hline
    \textbf{Title} & Configuration CI environment\\ \hline
    \textbf{Id} & s5\\ \hline
    \textbf{Estimated time} & 4h \\ \hline
    \textbf{Description} & To deliver high quality software a continuous integration environment is required. Following tools should be evaluated and configured for further use:
    \begin{itemize}
        \item Version control (git)
        \item Project build automation tool (gradle)
        \item continuous integration service (Travis CI)
    \end{itemize}
    \\
    \hline
    \end{tabular}


\section{Milestone two - m2}

\begin{itemize}
\item Closing date date: 22.10.2014
\item Available time: ca. 45h
\end{itemize}

    \begin{tabular}{ | p{10cm} | p{2cm} | p{2cm} |}
    \hline
    \textbf{Working package} & \textbf{Shortcut}& \textbf{Estimated time} \\ \hline
    Implementation test framework & s6 & 20h \\
    Prototype Integration of justext/boilerpipe & s7 & 17h \\
    Collection of test data & s8 & 8h \\ \hline
    Total &  & 45h\\
    \hline
    \end{tabular}


\subsection {Stories m2}


    \begin{tabular}{ | p{4cm} | p{10cm} |}
    \hline
    \textbf{Title} & Implementation Config Reader \\ \hline
    \textbf{Id} & s6\\ \hline
    \textbf{Estimated time} & 4h \\ \hline
    \textbf{Description} & The configuration for the test framework is located in a text file in the resources folder of the project. The data is formatted  in a key value structure. This text file is read at the startup of the program and saved in a Config object.\\ 
    \hline
    \end{tabular} \\\\


    \begin{tabular}{ | p{4cm} | p{10cm} |}
    \hline
    \textbf{Title} & Implementation File Reader \\ \hline
    \textbf{Id} & s7\\ \hline
    \textbf{Estimated time} & 6h \\ \hline
    \textbf{Description} & The html and content files are located in two folders (content/html) in the resources folder of the project.. For each file pair with the same name, a test object is generated and the content of the file is read and put into the test objects.\\ 
    \hline
    \end{tabular} \\\\


        \begin{tabular}{ | p{4cm} | p{10cm} |}
    \hline
    \textbf{Title} & Implementation File Writer \\ \hline
    \textbf{Id} & s8\\ \hline
    \textbf{Estimated time} & 4h \\ \hline
    \textbf{Description} & The results of a test is written to an output text file in the resources folder of the project.\\ 
    \hline
    \end{tabular} \\\\


            \begin{tabular}{ | p{4cm} | p{10cm} |}
    \hline
    \textbf{Title} & Implementation Test Manager \\ \hline
    \textbf{Id} & s9\\ \hline
    \textbf{Estimated time} & 6h \\ \hline
    \textbf{Description} & The Test Manager contains the business logic of the program and coordinates the reading, testing and writing.\\ 
    \hline
    \end{tabular} \\\\


    \begin{tabular}{ | p{4cm} | p{10cm} |}
    \hline
    \textbf{Title} & Prototype Integration of boilerpipe\\ \hline
    \textbf{Id} & s10\\ \hline
    \textbf{Estimated time} & 4h \\ \hline
    \textbf{Description} &  Implementation of a small prototype which uses the existing implementation of boilerpipe. A final interface for boilerpipe is defined for further use.\\ 
    \hline
    \end{tabular} \\\\

    \begin{tabular}{ | p{4cm} | p{10cm} |}
    \hline
    \textbf{Title} & Prototype Integration of justext\\ \hline
    \textbf{Id} & s11 \\ \hline
    \textbf{Estimated time} & 12h \\ \hline
    \textbf{Description} &  Implementation of a small prototype which uses the existing implementation of justext. A final interface for justext is defined for further use. \\ 
    \hline
    \end{tabular} \\\\


    \begin{tabular}{ | p{4cm} | p{10cm} |}
    \hline
    \textbf{Title} & Collection of test data\\ \hline
    \textbf{Id} & s12\\ \hline
    \textbf{Estimated time} & 8h \\ \hline
    \textbf{Description} &  To evaluate the functionality of the text extraction algorithms, a certain amount of test data is needed. This test data contains HTML files of several web pages. The HTML code is categorized into content and boilerplate.\\ 
    \hline
    \end{tabular} \\\\




\section{Milestone three - m3}

\begin{itemize}
\item Closing date date: 5.11.2014
\item Available time: ca. 30
\end{itemize}

    \begin{tabular}{ | p{10cm} | p{2cm} | p{2cm} |}
    \hline
    \textbf{Working package} & \textbf{Shortcut}& \textbf{Estimated time} \\ \hline
    Implementation test framework & s13 & 20h \\
    Final integration of justext / boilerplate & s10 & 10h \\ \hline
    Total &  & 30h\\
    \hline
    \end{tabular}

\subsection {Stories m3}

    \begin{tabular}{ | p{4cm} | p{10cm} |}
    \hline
    \textbf{Title} & Implementation text comparator analyzer \\ \hline
    \textbf{Id} & s13\\ \hline
    \textbf{Estimated time} & 12h \\ \hline
    \textbf{Description} & Implement the analyzer component. The content text and the extracted text by an algorithm needs to be compared and the values TP, FP, TN, FN need to be calculated.\\ 
    \hline
    \end{tabular} \\\\


    \begin{tabular}{ | p{4cm} | p{10cm} |}
    \hline
    \textbf{Title} & Implementation math analysis \\ \hline
    \textbf{Id} & s14\\ \hline
    \textbf{Estimated time} & 6h \\ \hline
    \textbf{Description} & After getting the values for TP, FP, TN, FN further values need to be calculated. These values are Precision, Recall, F-Measure and Fallout.\\ 
    \hline
    \end{tabular} \\\\


    \begin{tabular}{ | p{4cm} | p{10cm} |}
    \hline
    \textbf{Title} & Final implementation of justext and boilerpipe\\ \hline
    \textbf{Id} & s15\\ \hline
    \textbf{Estimated time} & 6h \\ \hline
    \textbf{Description} &  After implementing prototypes for both Justext and boilerpipe, they need to be implementend in the main application so that test cases can be classified by the algorithms.\\ 
    \hline
    \end{tabular} \\\\





\section{Milestone four - m4}

\begin{itemize}
\item Closing date date: 19.11.2014
\item Available time: ca. 30h
\end{itemize}

    \begin{tabular}{ | p{10cm} | p{2cm} | p{2cm} |}
    \hline
    \textbf{Working package} & \textbf{Shortcut}& \textbf{Estimated time} \\ \hline
    Evaluation environment for results & s11 &20h \\
    Research on new algorithm & s12 &10h \\ \hline
    Total &  & 30h\\

    \hline
    \end{tabular}

\subsection {Stories m4}

    \begin{tabular}{ | p{4cm} | p{10cm} |}
    \hline
    \textbf{Title} & Evaluation of results\\ \hline
    \textbf{Id} & s16\\ \hline
    \textbf{Estimated time} & 8h \\ \hline
    \textbf{Description} &  The test framework does produce a lot of output data. To investigate the results more closely, the data needs to be put into a form that allows specific information to be extracted. To do so, an excel sheet is needed, where the output data from the test framework can be imported into excel with ease.\\ 
    \hline
    \end{tabular} \\\\


        \begin{tabular}{ | p{4cm} | p{10cm} |}
    \hline
    \textbf{Title} & Improve the text analysis\\ \hline
    \textbf{Id} & s17\\ \hline
    \textbf{Estimated time} & 8h \\ \hline
    \textbf{Description} &  Right now the text analysis is not very precise. Mr. Kaufmann suggested to use a diff tool to compare the text files with each other. So a diff library needs to be found, tested and integrated into the existing application.\\ 
    \hline
    \end{tabular} \\\\

        \begin{tabular}{ | p{4cm} | p{10cm} |}
    \hline
    \textbf{Title} & Add a test case filter\\ \hline
    \textbf{Id} & s18\\ \hline
    \textbf{Estimated time} & 8h \\ \hline
    \textbf{Description} &  Investigating a specific test is not possible yet. To do so some kind of filter is needed. It should be possible to set a test case for inspection in the config file and as a result, an extra output file with information about this test case should be generated.\\ 
    \hline
    \end{tabular} \\\\

        \begin{tabular}{ | p{4cm} | p{10cm} |}
    \hline
    \textbf{Title} & Research on new algorithm\\ \hline
    \textbf{Id} & s19\\ \hline
    \textbf{Estimated time} & 20h \\ \hline
    \textbf{Description} &  A research should be done on the idea of the RSS algorithm. Based on this research it should be possible to decide, if we are going to implement it or not.\\ 
    \hline
    \end{tabular} \\\\


\section{Milestone five - m5}

\begin{itemize}
\item Closing date date: 8.12.2014
\item Available time: ca. 38h
\end{itemize}

    \begin{tabular}{ | p{10cm} | p{2cm} | p{2cm} |}
    \hline
    \textbf{Working package} & \textbf{Shortcut}& \textbf{Estimated time} \\ \hline
    Implementation of new algorithm & s13 & 19h \\
    Complete documentation & s15 & 15h \\
    Prepare final presentation & s15 & 4h \\ \hline
    Total &  & 38h\\
    \hline
    \end{tabular}

\subsection {Stories m5}

    \begin{tabular}{ | p{4cm} | p{10cm} |}
    \hline
    \textbf{Title} & Add filter function to test framework\\ \hline
    \textbf{Id} & s21\\ \hline
    \textbf{Estimated time} & 3h \\ \hline
    \textbf{Description} &  It is very time consuming to isolate interesting test cases of interest from the big set of test data. To simplify this process, a filter for the test cases is needed. It should be possible to filter the results for a certain algorithm for specific values like Precision or Recall. This filter values are set in the configuration file.\\ 
    \hline
    \end{tabular} \\\\

    \begin{tabular}{ | p{4cm} | p{10cm} |}
    \hline
    \textbf{Title} & Adapt the Justext algorithm for block extraction\\ \hline
    \textbf{Id} & s22\\ \hline
    \textbf{Estimated time} & 5h \\ \hline
    \textbf{Description} &  Right now only the calculated values for True Positive, Precision, etc. are available in the detailed result report. To investigate the results more closely, more information is needed which is the classification from the algorithms for the single HTML blocks. To do so, the Justext algorithm needs to be adjusted so that it is possible to get the classification information for each HTML block.\\ 
    \hline
    \end{tabular} \\\\

    \begin{tabular}{ | p{4cm} | p{10cm} |}
    \hline
    \textbf{Title} & Adapt the Boilerpipe algorithm for block extraction\\ \hline
    \textbf{Id} & s23\\ \hline
    \textbf{Estimated time} & 5h \\ \hline
    \textbf{Description} &  Right now only the calculated values for True Positive, Precision, etc. are available in the detailed result report. To investigate the results more closely, more information is needed which is the classification from the algorithms for the single HTML blocks. To do so, the Boilerpipe algorithm needs to be adjusted so that it is possible to get the classification information for each HTML block.\\ 
    \hline
    \end{tabular} \\\\


     \begin{tabular}{ | p{4cm} | p{10cm} |}
    \hline
    \textbf{Title} & Integrate the adapted algorithms into the main application\\ \hline
    \textbf{Id} & s24\\ \hline
    \textbf{Estimated time} & 6h \\ \hline
    \textbf{Description} &  After adapting the algorithms so that it is possible to extract the block classification information, the main application needs to be adapted that it can handle the additional block information and put it into the output file when needed.\\ 
    \hline
    \end{tabular} \\\\


    \begin{tabular}{ | p{4cm} | p{10cm} |}
    \hline
    \textbf{Title} & Complete documentation\\ \hline
    \textbf{Id} & s25\\ \hline
    \textbf{Estimated time} & 15h \\ \hline
    \textbf{Description} &  Complete and review all chapters of the documentation.  \\ 
    \hline
    \end{tabular} \\\\

    \begin{tabular}{ | p{4cm} | p{10cm} |}
    \hline
    \textbf{Title} & Prepare final presentation\\ \hline
    \textbf{Id} & s26\\ \hline
    \textbf{Estimated time} & 4h \\ \hline
    \textbf{Description} &  Prepare the final presentation and the final printed / digital version of the thesis.  \\ 
    \hline
    \end{tabular} \\\\





\section{Time tracking}
    \label{sec:Time tracking}

To keep track of the time, the tool toggl was used. Toggl is an easy to use time tracking tool. In the following table, all the performed tasks and the used time are listed.



TODO: put table here


