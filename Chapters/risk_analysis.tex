% Chapter Template

\chapter{Risk Analysis} % Main chapter title

\label{ChapterX} % Change X to a consecutive number; for referencing this chapter elsewhere, use \ref{ChapterX}

\lhead{\emph{Risk Analysis}} % Change X to a consecutive number; this is for the header on each page - perhaps a shortened title

%----------------------------------------------------------------------------------------
%	SECTION 1
%----------------------------------------------------------------------------------------

\section{Version}

\begin{tabular}{| p{1.5cm} | p{2cm} | p{9cm} | p{1.5cm} |}
	\hline
	Version & Date 		& Change & Author \\ \hline
	0.1 	& 20.09.2014 		& Setup document  										& JR \\ \hline
	0.2 	& 28.09.2014		& Add risks, evaluation, consequences					& JR \\ \hline
	1.0 	& 05.10.2014		& Grammar, layout 										& JR \\ \hline
	1.1 	& 8.10.2014			& Risk text comparison added  										& JR \\ \hline

\end{tabular}

\section{Introduction}

\subsection{Purpose}

This document evaluates and calculates all possible risks and defines actions that can minimize these risks as well as possible.

\section{Risk evaluation}

\subsection{Unclear requirements}

The start of a project is normally not an easy task because its requirements are only vaguely known. If they are not well defined as soon as possible, the requirements will stay vague throughout the whole project, which can lead to a disaster. 

\subsection{New technologies}

New technologies present in this project are:

 \begin{itemize}
    \item Gradle
    \item Travis CI
    \item Python
\end{itemize}

Each of them brings its own risk.

\subsection{Integration of Boilerpipe}

The Boilerplate algorithm needs to be integrated into the text extraction framework. Every interface of an external component is a possible risk factor.

\subsection{Integration of Justext}


The Justext algorithm needs to be integrated into the text extraction framework. Every interface of an external component is a possible risk factor.


\subsection{Implementation of a new algorithm}

The development and implementation of a new algorithm is predestined to generate risks.

\subsection{Comparison of text files}

It is not yet clear how the results of the algorithms should be compared to the correct content files. There are two possibilities. First: the comparison based on text, second: the comparison based on HTML blocks generated by the algorithms. 
The first approach could be difficult because defining, which words from the extracted files are content for sure and not some words from an advertisement which match the content words by chance.
The second approach could be difficult because it is possible that the algorithms generate different blocks as they are defined in the content file and it would be very hard do determine if a block is now classified correctly or not.

\section{Assessment of risks}

\begin{table}[h]
\begin{tabular}{|l|c|c|c|}
\hline
\textbf{Risk} & \textbf{Impact} & \textbf{Probability of occurrence} & \textbf{Risk factor} \\ \hline
Unclear requirements & 2 & 4 & 8\\ \hline
New technologies & 3 & 3 & 9 \\ \hline
Integration Boilerpipe & 5 & 1 & 5\\ \hline
Integration Justext & 5 & 5 & 25 \\ \hline
Implementation RSS algorithm & 1 & 5 & 5\\ \hline
Comparison text files & 5 & 5 & 25 \\ \hline
\end{tabular}
\end{table}

\section{Consequences}



\subsection{Unclear requirements}

As I am working with the supervisors every day, it is very easy to prevent misunderstandings by communicating with the client as soon as any difficulty appears. Nonetheless, misunderstandings can occur between student and expert. In order to prevent this, it is necessary to have a document defining the requirements as soon and as exact as possible. This will be done in the form of the system requirement specification in the first mile stone. Possible ambiguities can be clarified at the first milestone meeting.

\subsection{New technologies}

It is important to do prototyping with new technologies in the first phase of the project to eliminate these risks as soon as possible.

Gralde and Travis CI are needed in the first milestone to launch the programming environment. If there is any problem it will occur in a very early stage of the project and a possible solution can be found.

\subsection{Integration of Boilerpipe}

This risk is rated not that high because its implementation is in Java and it provides a Java API. Nevertheless, a prototype should be done as soon as possible to prevent any unwelcome surprises with the interface.

\subsection{Integration of Justext}
\label{subsec:Integration Justext}

This aspect is classified as the one of the highest risk of all. This is because the implementation happens in Python and it is not clarified yet how it will be integrated into the text extraction framework. An analysis of a possible solution with prototypes needs to be done as soon as possible.

Possible solution are:
\begin{itemize}
\item jython (\url{http://www.jython.org})
\item Implementation in Java
\item Java Processor Interface 
\end{itemize}


\subsection{Implementation of a new algorithm}

This risk has a very high probability of occurrence because it is very likely that a development and an implementation of a new algorithm will cause problems. There is no real solution to that risk. However, because this requirement is noncompulsory, the impact on the outcome of the project is very low. Furthermore, Patrik Lengacher, the tutor of this project, is very experienced in this subject area and will be able to assist if any problems occur.

\subsection{Comparison of text files}
This risk is classified as one of the highest risks as well since this problem needs to be solved with all certainty and there is no solution yet. Because of this, the problem needs to be discussed with the examiners and prototypes need to be tested as soon as possible.