% Chapter Template

\chapter{Approach} % Main chapter title

\label{ChapterX} % Change X to a consecutive number; for referencing this chapter elsewhere, use \ref{ChapterX}

\lhead{\emph{Approach}} % Change X to a consecutive number; this is for the header on each page - perhaps a shortened title

%----------------------------------------------------------------------------------------
%	SECTION 1
%----------------------------------------------------------------------------------------

This chapter describes the approach developing the application. Most of the content in this chapter is covered by the project documents as well. It describes the general idea how the application is developed. 


\section{Planning}

The planning of this project was a very straight forward task. Since it is a small one man project and the Eckdaten were clearly defined there was not a lot of spielraum.  However I am working part time for Layzapp and this project was not the only task during the semester and the project needed to be coordinated somehow with other tasks. My solution was to work about one to two weeks for on this project and switch to other task for the next one to two weeks. Switching between multiple projects within three days is not very effective because one needs always some hours to get back into the topic. 
A project plan was done for the whole time line of the project. Several milestones and its delivery objects were defined. The first milestone was then planned more closely with stories. After finishing the first milestone, the second one was planned more closely. So the vorgehen is a combination of the RUP vorgehensmodel and Scrum. Since it was a one man team no weekly scrum meetings were done but the milestones were defined as sprint meetings where the delivery objects were reviewed with the supervisors and tasks for the next milestone were defined or adjusted. During the project, the tool toggl was used as a time tracker and task manager. The whole planning as well as the time tracking auswertung can be found in the abbreviations \ref{sec:Chapters/planning}.

\section{Requirement Engineering}



\section{Knowledge}

I had no knowledge about text extraction at all. This was a big risk at the beginning at the project and needed to be handled somehow. I researched the topic in the internet and got help from my advisors as well and 



\section{Programming}



