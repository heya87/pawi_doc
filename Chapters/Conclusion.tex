% Chapter Template

\chapter{Discussion and Outlook} % Main chapter title

\label{Discussion and Outlook} % Change X to a consecutive number; for referencing this chapter elsewhere, use \ref{ChapterX}

\lhead{\emph{Discussion and Outlook}} % Change X to a consecutive number; this is for the header on each page - perhaps a shortened title

%----------------------------------------------------------------------------------------
%	SECTION 1
%----------------------------------------------------------------------------------------

\section {Conclusion}

With the test framework it is now possible to compare text extraction algorithms against each other and get a general information about their performance. We can now rate the performance of the two algorithms JusText and Boilerpipe and with the implemented software architecture it is easy to integrate and test additional algorithms as well. We can not only compare algorithms with each other but as well investigate the single algorithms and their output more closely for specific test cases. This helps to find weaknesses and strengths of existing algorithms and may help to improve them or implement our own algorithm. Based on the results, we can say that JusText performs better context extraction than Boilerpipe.
The missing requirement is the implementation of a new algorithm but by dropping this requirement it was possible to implement the additional functions for investigating the existing algorithms more closely, which will help to improve the existing ones as well as developing a possible new approach in the future.

\section{Lessons learned}

\subsection{Planning}

We underestimated the effort needed for the documentation. Especially putting together all the single documents and writing the main part took more time than we expected. There are a lot of little things which we did not took in consideration during the planning phase but still need to be done at some point.

\subsection{Programming}
\begin{itemize}
\item It is very important to test software early and often. Most of the critical components were tested right from the beginning. However, if they where not, a time loss was almost guaranteed because bugs sneak into the code and needed to be taken care of.

\item I used the programming language Python the first time for this project. Although the JusText project needed only small adjustments, I have gained a basic knowledge about Python and implemented a solution with which it is possible to interact with Python from a Java application.

\item I had very basic knowledge about text processing and text extraction. Working with the topic for this project extended my knowledge quite a bit.


\end{itemize}

\section{Further work}

With the test framework we have now a tool to examine text extraction algorithms very close. The next steps will be an exact evaluation of the weaknesses of the algorithms so we can improve them or we can implement our own approach, which would need to handle the these problems better. The test framework can then be used at any point during the development phase to check the performance of the algorithm very easy and compare it with the existing ones.  


