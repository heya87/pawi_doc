% Chapter Template

\chapter{Conclusion} % Main chapter title

\label{Conclusion} % Change X to a consecutive number; for referencing this chapter elsewhere, use \ref{ChapterX}

\lhead{\emph{Conclusion}} % Change X to a consecutive number; this is for the header on each page - perhaps a shortened title

%----------------------------------------------------------------------------------------
%	SECTION 1
%----------------------------------------------------------------------------------------

\section {Conclusion}

With the test framework it is now possible to compare text extraction algorithms against each other and get a general information about the performance. We can now classify the performance of the two algorithms Justext and Boilerpipe and with the implemented software architecture it is easy to integrate and test additional algorithms as well. We can not only compare algorithms with each other but as well investigate the single algorithms and their output more closely for specific test cases. This helps to find problems and strengths of existing algorithms and may help to improve them or implement our own algorithm. So it can be told that all the mandatory requirements are fulfilled and with the implementation of the detailed analysis there are even more functions available. 
The missing requirement is the implementation of the RSS algorithm but with dropping this requirement it was possible to implement the additional functions for investigating the existing algorithms more closely, which will help to improve the existing ones as well as developing a possible new approach.

\section{Lessons learned}

\subsection{Planning}

I did underestimate the effort needed for the documentation. Especially putting together all the single documents and writing the main part took more time then i expected. There are a lot of little things which I did not took in consideration during the planning phase but which need to be done at some point.

\subsection{Programming}
\begin{itemize}
\item I realized once more that it is very important to test software early and often. I tried to test all the critical components right from the beginning. However if I did not, a time loss was almost expected for sure because bugs sneaked into the code and as a result, I was searching for the problems on the wrong places. 

\item I used the programming language Python the first time for this project. Even I only had to do small adjustments on the Justext project , I now know the basic syntax as well as how to interact with it from a Java application.

\item I had very basic knowledge about text processing and text extraction. Working with the topic for this project extended my knowledge quite a bit.


\end{itemize}

\section{Further work}

With the test framework we have now the tool to examine text extraction algorithms very close. The next steps will be an exact evaluation of the weaknesses from the algorithms so we can improve them or we can implement our own approach, which handles these problems better. The test framework can then be used at any point during the development phase to check the performance of the algorithm very easy and compare it with the existing ones.  


