% Chapter Template

\chapter{MS4 meeting report} % Main chapter title

\label{ChapterX} % Change X to a consecutive number; for referencing this chapter elsewhere, use \ref{ChapterX}

\lhead{\emph{MS4 meeting report}} % Change X to a consecutive number; this is for the header on each page - perhaps a shortened title

%----------------------------------------------------------------------------------------
%	SECTION 1
%----------------------------------------------------------------------------------------

\section{Introduction}

This document is a short report about the MS4 meeting. The meeting took place on the 21.11.2014.

\section{Version}


\begin{tabular}{| p{1.5cm} | p{2cm} | p{9cm} | p{1.5cm} |}
    \hline
    Version 	& Date      		& Change & Author 								\\ \hline
    0.1    		& 21.11.2014        & Setup document        				& JR 	\\ \hline
    1.0 		& 21.11.2014 		& add meeting report 					& JR 	\\ \hline
\end{tabular}


\section{Attendees}
\begin{itemize}
\item Joel Rolli
\item Michael Kaufmann
\item Patrick Lengacher
\end{itemize}

\section{Delivery objects}

\begin{itemize}
\item Evaluation environment for output data of test framework

\item First approach to new algorithm

\item Interface definition for justext/boilerplate components
\item HTML test data
\end{itemize}


\section{Decisions}

All delivery objects are approved. A first draft of the final documentation was reviewed
by Mr. Kaufmann. The structure is good in general. However some chapters are
renamed and it was decided that the project documents belong into the appendix and
are not integrated into the main report.
Furthermore, the statistical results which were produced with the application were re-
viewed and some parts of the source code were presented. We decided that implementing
an extra algorithm is too time consuming for the time left and that the focus for the
last milestone is on the documentation, the analysis of the existing algorithms and im-
provement of the existing work.
One more point was discussed concerning the javadoc. There is no need to do javadoc
for each method. It is ok to do javadoc for the important classes and methods.
We decided that the final presentation is held on the 11. December if everyone (Michael
Kaufmann, Patrick Huber, Patrick Lengacher, Manuel Schneider) is free on this day.

\section{Rework}

No rework needs to be done.