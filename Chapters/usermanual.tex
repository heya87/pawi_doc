% Chapter 2

\chapter{User manual} % Main chapter title

\label{User manual} % For referencing the chapter elsewhere, use \ref{Chapter1} 

\lhead{User manual} % This is for the header on each page - perhaps a shortened title

%----------------------------------------------------------------------------------------

\section{Introduction}


\section{System requirements}

To use the text extraction test framework, the system needs to fulfill following system requirements.

\begin{tabular}{| p{2cm} | p{1.5cm} | p{9.5cm} |} 
	\hline
	\textbf{Software} & \textbf{Version} & \textbf{Source} \\ \hline
	Ubuntu & 12.04 & \url{http://releases.ubuntu.com/12.04/} \\ \hline
	git & 2.2.0 & \url{http://git-scm.com/download/linux} \\ \hline
	python & 2.7.6 & \url{https://www.python.org/download/releases/2.7.6/} \\ \hline
\end{tabular}


\section{Installation}

\subsection{}
\begin{enumerate}

\item Checkout the repository with git to your favorite directory.
\begin{lstlisting}
git clone https://github.com/heya87/pawiTwo
\end{lstlisting}


\item Use your favorite console application and navigate into the source directory of the checked out repository.

\item Build the project with the gradle wrapper.
\begin{lstlisting}
./gradlew build
\end{lstlisting}



\item Put the test files you would like into the following folders

\item Put the content files into following folder: ~/build/resources/main/content
\item Put the HTML files into following folder: ~/build/resources/main/html
\item Run the test with the gradle wrapper.

\begin{lstlisting}
./gradlew build
\end{lstlisting}
\item Find the results in the output folder: ~/build/resources/main/


\end{enumerate}


\section{Usage \& Configuration}

